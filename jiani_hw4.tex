\documentclass[12pt]{article}
%[10pt,technote]{IEEEtran}
\usepackage{hyperref}
\usepackage{graphicx}
\usepackage[affil-it]{authblk}
\usepackage{color}
\usepackage{amsgen,amsmath,amstext,amsbsy,amsopn,amssymb}
\usepackage{geometry}
\usepackage{subcaption}
\usepackage{caption}
\usepackage{wrapfig}
\usepackage{comment}
\usepackage{mathrsfs}
\usepackage{upgreek}
\usepackage{amssymb}
\usepackage{textcomp}
\usepackage{amsmath}
\usepackage{tcolorbox}
\usepackage{listings}
\usepackage{fancyhdr}
\usepackage{cite}
\usepackage{wasysym} 
%\usepackage{subfigure}
%\usepackage{wraptable}

\geometry{left=2.5cm, right=2.5cm,top=2.5cm,bottom=2.5cm}



\setcounter{secnumdepth}{0}
\title{\bf Homework 4}
\author{Jiani Gao\\jgao30@binghamton.edu}
\affil{Department of Economics, Binghamton University}
\date{\today}
\pagestyle{fancy}
\chead{Econ634 hw4} \lhead{}\rhead{}
\cfoot{jgao30@binghamton.edu}

\begin{document}
\maketitle
\newpage
\tableofcontents	

\newpage

\section{Step 1: Frim's problem}

Firm's problem now is:
\[\max\limits_{\{K_{t+1}^d,N_t^d\}}\sum\limits_{t=0}^\infty(\frac{1}{\prod_{i=0}^{t} r_i})\pi (K,N;w_t,r_t)=\max\limits_{\{K_{t+1}^d,N_t^d\}}\sum\limits_{t=0}^\infty(\frac{1}{\prod_{i=0}^{t} r_i})(K_t^\alpha N_t^{1-\alpha}-w_t N_t-r_t K_{t}+(1-\delta)K_t) \tag{1}
\]	
By doing the first order condition with respect to $K_{t+1}$ and $N_t$, we can get the following conditions:

\[\alpha K_{t+1}^{\alpha -1}N_{t+1}^{1-\alpha}=r_{t+1}+\delta-1
\tag{2}
\]
\[(1-\alpha)K_t^\alpha N_t^{-\alpha}=w_t\tag{3}
\]
Equation (2) also holds for period t, so we should also have that 
\[\alpha K_{t}^{\alpha -1}N_{t}^{1-\alpha}=r_{t}+\delta-1
\tag{4}
\]
\section{Step 2: Household value function}
The household choose $c_t$ and $a_{t+1}$ to maximize the expected utlity, or we can write $c_t$ as a function of $a_{t+1}$ so that we can write down the value function as:
\[v(z_t,a_t)=\max\limits_{a_{t+1}}U(c_t)+\beta  E v(z_{t+1},a_{t+1})\tag{5}
\]
where $c_t=z_tw_t\bar{l}+r_ta_t-a_{t+1}$, and the utility function is CRRA utility function.

\section{Step 3: Discretize the exogenous state variable}
For $\rho=0.5$, $\sigma_\epsilon=0.2$ and number of z equals to 5, we can get a discrete set of possible values for z:
\[z=\begin{bmatrix}
0.5002&0.7072&1.0000&1.4140&1.9993
\end{bmatrix}\tag{6}
\]
we can also get the invatiant distribution from $m\times m$ transition matrix:
\[\pi^{inv}(z)=\begin{bmatrix}
0.0145&0.2189&0.5333& 0.2189& 0.0145

\end{bmatrix}\tag{7}
\]
So the aggregate labor supply $N^s=1.0338$
\section{Step 4: Discretize the endogenous state variable}
In this step, we discretize a into a grid with 500 points. At this time, I choose $a_{max}=5$, and we will adjust this number later according to the binding constraint.
\section{Step 5: Solving the model numerically}
From Step 3, we know that $N^s=1.0338$, thus $N_d=1.0338$ due to market clear condition. Now let's guess $K=2$, and using equation (3) and equation (4), we can calculate the value for $r$ and $w$.\\\\
\[r_{t}=\alpha K_{t}^{\alpha -1}N_{t}^{1-\alpha}+1-\delta \tag{8}\]
\[w_t=(1-\alpha)K_t^\alpha N_t^{-\alpha} \tag{9}\]
\noindent
Please see the Appendix for Matlab code.

\section{Step 6: Analyze the results}
\begin{enumerate}


	\item [a.] The steady state interest rate I get is $r=1.0928$, while in the complete market case $r^{CM}=1/\beta=1.0101$. So interest rate in the incomplete market is higher. Why???
    \item [b.] Below is the policy functions for m productivity series:
    \begin{figure}[!h]\centering
	\includegraphics[width=6in,height=5in]{policy}
    \end{figure}
    \item [c.]This time we take asset holding as wealth(why?), and the Lorenz curve is:
    \begin{figure}[!h]\centering
    	\includegraphics[width=5in,height=5in]{wgini}
    \end{figure}
    \newline The gini index is smaller comparing with gini index in Hugget model.
\end{enumerate}
\section{Step 7: Alternative way of coding}
Solving the value function on a coarse grid, and using the result as a starting value. Then use policy function iteration.\\\\
\\\\
Please see appendix.

\section{Appendix: Matlab code}
    \subsection{Way 1}
     \begin{verbatim}
    alpha=1/3;
    beta=0.99;
    s=2;
    delta=0.025;
    
    
    %%%%%question 3%%%%%%%%%%%%%%%%%%%%%%%%%%%%%
    m=5;
    rho=0.5;
    sigma=0.2;
    d=3;
    
    [z,zprob] =TAUCHEN(m,rho,sigma,d);
    z=exp(z);
    pi=zprob^1000;
    pz=pi(1,:);
    ns=pz*z;
    %%%%%%%question 4%%%%%%%%%%%%%%%%%%%%%%%%%%%%
    a_min=0;
    a_max=300; 
    a_num=100;
    a=linspace(a_min,a_max,a_num);
    %%%%%%question 5%%%%%%%%%%%%%%%%%%%%%%%%%%%%%
    nd=ns;
    k_max=100;
    k_min=50;
    d=1;
    while d>=0.001;
    
    k_guess=(k_max+k_min)/2;
    r=alpha*k_guess^(alpha-1)*nd^(1-alpha)+1-delta;
    w=(1-alpha)*k_guess^alpha*nd^(-alpha);
    
    %consumption fn
    cons=bsxfun(@minus,r*a',a);
    cons=bsxfun(@plus,cons,permute(w*z,[2 3 1]));
    %return fn
    ret=cons.^(1-s)/(1-s);
    ret(cons<0)=-Inf;
    %value fn and policy fn
    v_guess=zeros(m,a_num);
    v_tol = 1;
    while v_tol >.0001;
    % CONSTRUCT RETURN + EXPECTED CONTINUATION VALUE
    
    vf=bsxfun(@plus,ret,permute(beta*zprob*v_guess,[3,2,1]));
    
    % CHOOSE HIGHEST VALUE (ASSOCIATED WITH a' CHOICE)
    [vfn,pol_indx]=max(vf,[],2);
    v_tol=max(abs(vfn(:)-v_guess(:)));
    v_guess=shiftdim(vfn,2);
    end;
    % KEEP DECSISION RULE
    pol_indx=permute(pol_indx, [3 1 2]);
    pol_fn=a(pol_indx); 
    % SET UP INITITAL DISTRIBUTION
    Mu=ones(m,a_num)/(m*a_num);
    
    % ITERATE OVER DISTRIBUTIONS
    m_tol=1;
    while m_tol>0.0001
    [z_ind, a_ind, mass] = find(Mu > 0); % find non-zero indices
    
    MuNew = zeros(size(Mu));
    
    
    for ii = 1:length(z_ind)
    apr_ind = pol_indx(z_ind(ii), a_ind(ii)); % which a prime does the policy fn prescribe?
    MuNew(:, apr_ind) = MuNew(:, apr_ind) +(zprob(z_ind(ii),:)*Mu(z_ind(ii),a_ind(ii)))';
    % which mass of households goes to which exogenous state?
    end
    m_tol=max(max(abs(MuNew-Mu)));
    Mu=MuNew;
    end     
    
    aggsav=Mu(:).*pol_fn(:);
    d=abs(k_guess-aggsav);
    if k_guess>aggsav
    k_min=k_guess;
    end
    if k_guess<aggsav;
    k_max=k_guess;
    end
    d;
    end
     
     %%%%%%%%%question 6
     plot(a,pol_fn(1,:),a,pol_fn(2,:),a,pol_fn(3,:),a,pol_fn(4,:),a,pol_fn(5,:)),legend('z1','z2','z3','z4','z5');
     
     %gini
     p=[Mu(1,:);Mu(2,:);Mu(3,:);Mu(4,:);Mu(5,:)];
     wealth=[a;a;a;a;a];
     wg=gini(p,wealth,true);
     title(['wealth gini index=',num2str(wg)]);
     
     
     \end{verbatim}
     \subsection{way 2}
     \begin{verbatim}
     %%%%%%%%%parameters%%%%%%%%
     alpha=1/3;
     beta=0.99;
     s=2;
     delta=0.025;
     k=30;
     
     %%%%%%%%%%%%%%%%%%%%%%%%%%%%%%%%%
     m=5;
     rho=0.5;
     sigma=0.2;
     d=3;
     
     [z,zprob] =TAUCHEN(m,rho,sigma,d);
     z=exp(z);
     pi=zprob^1000;
     pz=pi(1,:);
     ns=pz*z;
     %%%%%%%%%%%%%%%%%%%%%%%%%%%%%%%%%%%
     a_min=0;
     a_max=300;
     a_num=100;
     a=linspace(a_min,a_max,a_num);
     
     r=alpha*k_guess^(alpha-1)*nd^(1-alpha)+1-delta;
     w=(1-alpha)*k_guess^alpha*nd^(-alpha);
     
     %consumption fn
     cons=bsxfun(@minus,r*a',a);
     cons=bsxfun(@plus,cons,permute(w*z,[2 3 1]));
     %return fn
     ret=cons.^(1-s)/(1-s);
     ret(cons<0)=-Inf;
     %value fn and policy fn
     v_guess=zeros(m,a_num);
     
     % CONSTRUCT RETURN + EXPECTED CONTINUATION VALUE
     v_tol=1;
     
     while v_tol>1e-8
     
     vf=bsxfun(@plus,ret,permute(beta*zprob*v_guess,[3,2,1]));
     % CHOOSE HIGHEST VALUE (ASSOCIATED WITH a' CHOICE)
     [vfn,pol_indx]=max(vf,[],2);
     pol_indx=shiftdim(pol_indx,2);  
     pol_fn=a(pol_indx);  
     vfn=shiftdim(vfn,2);
     
     Q = makeQmatrix(pol_indx,zprob);
     u=bsxfun(@minus,r*a,pol_fn);
     u=u+repmat(z*w,[1 a_num]);
     ret=u.^(1-s)/(1-s);
     ret(u<0)=-Inf;
     
     u=ret(:);
     w_vec=v_guess(:);
     for ii=1:k
     w_new=u+beta*Q*w_vec;
     w_vec=w_new;
     end
     
     v_guess=w;
     
     v_tol=max(abs(vfn(:)-v_guess(:)));
     end
     
     %%%%%%%%%question 6
     plot(a,pol_fn(1,:),a,pol_fn(2,:),a,pol_fn(3,:),a,pol_fn(4,:),a,pol_fn(5,:)),legend('z1','z2','z3','z4','z5');
     
     %gini
     p=[Mu(1,:);Mu(2,:);Mu(3,:);Mu(4,:);Mu(5,:)];
     wealth=[a;a;a;a;a];
     wg=gini(p,wealth,true);
     title(['wealth gini index=',num2str(wg)]);
     
     
     \end{verbatim}

%\bibliographystyle{unsrt} 
%\bibliography{}
\end{document}
